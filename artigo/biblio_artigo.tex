\begin{thebibliography}{9}
% \bibitem{} XXXXXX, Y S.  %\emph{Titulo.}, Editora. data.

\bibitem{brasil1} BRASIL.  \emph{Constituição (1988). Constituição da República Federativa do Brasil: promulgada em 5 de outubro de 1988.}, Editora São Paulo: Saraiva. 2007. 440 p. (Coleção saraiva de legislação).

\bibitem{brasil2} BRASIL.  \emph{DECRETO Nº 4.887 (2003). Regulamenta o procedimento para identificação, reconhecimento, delimitação, demarcação e titulação das terras ocupadas por remanescentes das comunidades dos quilombos de que trata o art. 68 do Ato das Disposições Constitucionais Transitórias.}, 2003.

\bibitem{freire} FREIRE, Paulo.  \emph{Pedagogia do oprimido.}, 17ª ed. Rio de Janeiro, Paz e Terra. 1987.

\bibitem{gramsci} GRAMSCI, Antonio.  \emph{Cadernos do Cárcere. v. 2.}, Edição e tradução: Carlos Nelson Coutinnho. Rio de Janeiro: Civilização Brasileira, 2000. (p. 32 a 53)

\bibitem{mec} MEC, \emph{Educação Quilombola - Apresentação},  Atlas do Desenvolvimento Humano no Brasil (2013), Disponível em 
\url{http://portal.mec.gov.br/index.php?option=com_content&view=article&id=12396:educacao-quilombola-apresentacao&catid=321:educacao-quilombola&Itemid=684}.

\bibitem{muranakaeminto} MURANAKA, Maria Aparecida Segatto. MINTO, César Augusto.  \emph{Organização da Educação Escolar.},  In: OLIVEIRA, Romualdo Portela de; ADRIÃO, Theresa. Gestão, financiamento e direito à Educação: análise da Constituição Federal e da LDB. 3. ed. São Paulo: Xamã, 2007. (p. 43-62)

\bibitem{marta} OLIVEIRA, Marta Kohl de. \emph{VYGOTSKY – Apredizado e desenvolvimento: Um processo sócio-histórico.
}, Brasília, 2012.

\bibitem{illich} ILLICH, Ivan. 1926-129s.  \emph{Sociedade sem escolas: trad. de Lúcia Mathilde Endlich Orth.}, Petrópolis, Vozes. 1985. 188p. (educação e tempo presente, 10).

%\bibitem{wesley} OLIVEIRA, Wesley da Silva.  \emph{Quilombo Mesquita: Cultura, Educação e Organização Sociopolítica na construção do pesquisador coletivo / Wesley da Silva Oliveira.
%}, Editora Scipione. 1993.





\end{thebibliography}