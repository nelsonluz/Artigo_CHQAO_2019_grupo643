\maketitle
\begin{abstract}
\noindent 

\end{abstract}

\section*{Introdução}
\hspace{1.5cm}
Coloque aqui a introdução\\


\section*{Comando e Controle}
\hspace{1.5cm}
Coloque aqui o texto\\


\section*{Centro de Operações de Comando e Controle}
\hspace{1.5cm}
Coloque aqui o texto\\


\section*{Sistema de Comando e Controle - \textbf{\textit{Pacificador Web}}}
\hspace{1.5cm}
O Pacificador Web é um sistema de Comando e Controle destacado em operações de Garantia da Lei e da Ordem (GLO) e Grandes eventos. Seu emprego, deste 2012, na Rio+20, Copa das Confederações, Copa do Mundo, Olimpíadas Rio, além de diversas operações em todo território Nacional no acompanhamento dos comboios das autoridades (dignatários, delegações, etc) e o tratamentos de incidentes, que pode vim a colocar a operação em risco, a realizações das atividades por meios da matriz de sincronização.\\

\hspace{1.5cm}
O Pacificador fundamenta-se no conceito de um Centro de Operações (COp), composto por operadores nas suas estruturas física e por agentes móveis. Os agentes móveis são integrantes do COp realizando tarefas diversas, como comboios, varreduras, segurança de instalações e escoltas. Eles tem consigo um telefone celular androide, com o sistema móvel do Pacificador instalado,  enviando sua localização através da ligação por rede 3G/4G ao Centro interligado, onde pode enviar relatos de situação, ocorrências e realizar ações a ele designadas na matriz. Outros agentes podem esta dotados de rádio, do sistema troncalizado, com suas localizações enviada através do Mups, Motorola, sendo que neste o Centro só terá visualização das localizações dos agentes no terreno. \\

\hspace{1.5cm}
Nas sede cada operador realiza o acompanhamento em tempo real dos relatos, ocorrências, localizações, e as realizações das ações de cada agente móvel, há muitas vezes uma tela de maiores dimensões ou um (ou mais) videowall, com a finalidade de mostrar o cenário, com diversas localizações dos agentes, georreferenciado e em maior proporção destas informações. No Teatro de Operações pode existir mais de um COp, para cumprir a missão. O Pacificador Web compartilham continuamente, entre os Centros, informaçṍes relevantes que dar ao COp superior a visão e a capacidade de assumir a responsabilidade sobre as ações que deverão tomar os membros dos COps subordinados, além de delegar ações para os mesmos pela matriz de sincronização. Isto, gera a construção da Consciência Situacional Compartilhada, entre subordinados e decisor, no apoio da condução da operação.\\

Dentro do Sistema os agentes móveis foram divididos de acordo com o modo de operação, o qual caracteriza a função do agente naquele momento. As principais função são modo de agente de segurança, comboio, batedor, pontos de segurança, embarcações e escolta aérea. Os oficiais de ligação que geralmente é responsáveis por acompanhar as autoridades durante todos os deslocamentos e atividades oficiais, utiliza o modo comboio e agente de segurança respectivamente. O modo batedor é destinado aos batedores dos diversos sistemas de segurança que estiverem apoiando o Centro. As tropas que realizada todos a segurança em diversos pontos especificos durante a GLO ou Grade evento usa o simbolo ponto de segurança. Já o modo de embarcação é utilizado para navios em geral e os helicópteros, basicamente dos BAvEx, é identificado pelo modo de escolta aérea. Todos os agentes móveis, sejam com rádios trocalizados ou smartphone, nos diversos modo de operação, transmite suas localizações que são replicada para os servidores, para que todos os COps, entre eles o COTer e Ministério da Defesa, alcance a consciência situacional. As transmissão das localizações pelos agentes em campo é realizada sem a intervenção do mesmo, bastando para isto está conectado aos uma dos servidores.

\hspace{1.5cm}
A implementação do Pacificador Web baseá-se nos conceitos técnicos, tais como: Consistência no COp, Consistência entre COps e Tolerância a Partição. A Consistência no COp se dar por sabemos que dentro de um mesmo Centro os diversos operadores buscam soluções para os mesmos conjuntos de problemas. Consequentemente, é relevante que a visão operacional seja comum para cada operador, bem como para o decisor, no cenário de arquitetura cliente-servidor. Na busca por uma Consciência Situacional Compartilhada entre os vários Centros de Operação, as informações de COps diferentes deve ser sincronizados. A Consistência entre COps se dará com o usuário acessando um mesmo servidor ou através de uma replicação de dados entre servidores. Com relação a Tolerância a partição o sistema deve ser to flexível a falhas de rede. O Pacificador está configurado em um pool de sítios em cidades diferentes para que cada Centro continue funcionando mesmo falta de contato entre alguns e quando a rede é recuperação os dados são sincronizados entre os servidores.\\

\hspace{1.5cm}
Implantação do Sistema Pacificador, que durante a Rio+20 era realizada nos 17 (dezessete) Centros de Operações na cidade do Rio de janeiro/RJ sofreu uma mudança significativa buscando atender a todos território Nacional. Na preparação para a Copas das Confederações no ano de 2013, foram realizado a implantação dos pool de servidores na cidade Brasília/DF e Rio de Janeiro. As cidades são contingências entre si, além de prover redundância em caso de falha, a existência destes dois pool é justificado pela necessidade de um atendimento nas constante na cidade do Rio de Janeiro. Os servidores localizados em Brasília/DF busca atender as demandas dos COps em todo o Brasil, pela EBNet.\\

\section*{Analise de Dados do Sistema Pacificador Web}
\hspace{1.5cm}
Coloque aqui o texto\\


\section*{Conclusão}
\hspace{1.5cm}
Coloque aqui o texto\\