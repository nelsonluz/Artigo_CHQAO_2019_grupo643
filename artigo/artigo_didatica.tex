\maketitle

\begin{abstract}
\noindent O texto traz uma abordagem da educação Quilombola, dentro do contexto cultural e social. Onde o saber tradicional é a base a criação de uma identidade cultural, dos jovens da comunidade quilombola. Para isto é necessário pensar uma nova formar de fazer educação, junto com a sociedade local, em processo de troca entre educadores e alunos dentro de uma autonomia.
\end{abstract}


\section*{Introdução}
\hspace{1.5cm}
A criança ao adentrar nos sistema educacional, têm como meta seu "aprendizado" e o "desenvolvimento". Mas o que é estes pois processos. Segundo o dicionário: aprendizado, é o processo de adquirir conhecimentos pelo estudo, dentro de uma imitação  processual;e desenvolvimento é o ato ou efeito de desenvolver e também o crescimento ou expansão gradual. Temos nestas duas definição o conhecimento e o crescimento, como farol para a formação para o sujeito individual e coletivo. Mas como se construir este processo na educação atual. Alguns autores propõe novas formas de pensar-lá. Temos neste processo a questão  da autonomia, no contexto social, cultural.\\

\hspace{1.5cm}
Segundo MURANAKA e MINTO\cite{muranakaeminto}, a educação dentro conceito abrangente implica na necessidade de articular as diversas áreas que compõem os direitos sociais, cujo atendimento tem a ver com o grau de humanidade e cidadania que se deseja alcançar. No que abrange o ensino é mais restrito, sobretudo se lembrarmos que é só escolar, refere-se ao ensino ministrado na instituição escola. Mesmo que os dois, no cotidiano, sejam usados como sinônimos, eles constitui numa impropriedade nos dispositivos legais. Para GRAMSCI \cite{gramsci}, os grupos sociais cria para si, organicamente, algumas camadas de intelectuais com homogeneidade e consciência da própria função mas também no social e político. Os intelectuais "orgânicos" gera consigo uma nova classe, elaborando aspectos do tipo social novo.\\

\section*{Educação Cultural}
\hspace{1.5cm}
Nos primeiros momentos, o aprendizado era concedido pelo comportamentalismo e seus condicionamento: respondente e operante. Onde o primeiro é dado pelo pareamento de ações  e seus estímulos neutros. Já o segundo o reforço é dado em comportamento não automático, operante, por reforço positivo. Este tipo de processo é até idealizado no sistema educacional em diversos locais.  Hoje, os educadores baseados em vários pensadores como Paulo Feire e Vygotsky, Wallon e outros é pensado uma nova forma do fazer educação.\\

\hspace{1.5cm}
Segundo Vigotsky\cite{marta}, o educador é o mediador no processo de aprendizado da criança. para ele a mediação é um conceito central para compreensão sobre o funcionamento psicológico. Neste caso, o desenvolvimento se dar pelo processo de maturação do organismo individual, onde o aprendizado possibilita o despertar de processos internos de desenvolvimento do indivíduo com um certo ambiente cultural. Os educadores tem como papel, o avanço da compreensão de mundo dos alunos. Isto se dar a partir do conhecimento  dos alunos e sua busca pelos desenvolvimentos não alcançados. A escola sofre influencia nos pensamentos de Vygotsky e, assim, interferir na zona de desenvolvimento proximal dos alunos.\\

\hspace{1.5cm}
Para Illich\cite{illich} é necessário realizar a desescolização, com o meio-ambiente físico tornando acessível e os recursos físicos de aprendizagem reduzidos a instrumentos de ensino e disponível a todos em uma aprendizagem auto-dirigida e compartilhada. Diferente de um meio com os materiais educativos manipulados pelas escolas, dentro empacotamento do conhecimento pela industria. Neste processo o professor tem como instrumento de trabalho o livro-texto, que define suas metas. O intercâmbio de habilidades é uma forma de realizar este processo. Neste caso as habilidades comuns entre dois individuas (professor e educando) é a base ou ponto de partida para construção de conhecimento. Formando, assim, o compartilhar de conhecimento social e cultural, pelas habilidades comprovadas.\\

\hspace{1.5cm}
De acordo com Freire\cite{freire}, no seu método, o processo cultural revive  a vida em profundidade crítica. Neste processo a consciência tem sua formação nos projetos humanos e de seu mundo vivido. Tendo como suporte a cultural, que se apresenta como agregador de conhecimento e transformador do processo educacional. A cultura pode ser contada, mas gerar uma identificação do individuo, tem que ser vivida nas etapas da educação. Neste contexto o temos na comunidade do Quilombo Mesquita um processo de criação cultural, na busca de um identidade social.

\section*{Educação Quilombola}
\hspace{1.5cm}
A Escola Municipal Aleixo Pereira Braga I, localizada na Comunidade Quilombola do Mesquita. Esta comunidade se localizada no município Cidade Ocidental no Estado de Goias. Distante 35 km da cidade de Brasília no Distrito Federal. A escola é composta de 08 (oito) salas de aulas, 01 (uma) sala para secretaria e coordenação pedagógica, 01 (uma) sala de informática, 01 (uma) quadra de esporte e 01 (um) pátio coberto. O seu funcionamento se dar com 14 professores, sendo que 75\% (setenta e cinco por cento) são da comunidade. A diretora e a Coordenadora Pedagógica são indicação da prefeitura municipal. No seu corpo discente temos um total de 391 (trezentos e noventa e um) estudantes da comunidade e das fazendas da região. A escola atende o ensino fundamental, do 1º ao 9º ano, no período vespertino. O objetivo desta analise é ver como se dar o ensino tradicional, junto com o ensino regular, na manutenção de suas tradições culturais.\\

\hspace{1.5cm}
De acordo com o Decreto 4.887/2003, os quilombos são: grupos étnico-raciais segundo critérios de auto-atribuição, com trajetória histórica própria, dotados de relações territoriais específicas, com presunção de ancestralidade negra relacionada com a resistência à opressão histórica sofrida” \cite{brasil2}. A Lei 10.639/2003 altera a Lei 9394/1996 Lei de Diretrizes e Bases, nº 9394/96, e instituiu no Brasil um marco legal para que se inclua no currículo oficial das redes de ensino a obrigatoriedade da temática "História e Cultura Afro-Brasileira". Esta lei apesar de não ser específica para quilombos apresenta a possibilidade de se construir propostas de escolarização para quilombolas, pois não se pode falar de história e cultura afro-brasileira sem abordar a formação dos quilombos. A Constituição Federal de 1988 passou a reconhecer a legitimidade de posse e propriedade de terras consideradas Quilombos:\\
\begin{quote}
"Aos remanescentes de comunidades de quilombos que estejam ocupando suas terras é reconhecida a propriedade definitiva, devendo o estado emitir-lhes os títulos respectivos" \cite{brasil1}
\end{quote}
\hspace{1.5cm}
O termo “quilombo” vem das palavras “kilombo” da língua Quimbudo e “ochilombo” da língua Umbundo. No Brasil, a palavra tomou uma nova dimensão: comunidade de escravos fugitivos. Elas eram forma de resistência, com alguns aproximadamente até 20 mil habitantes. Tinha para seus habitantes a função, principal, de se esconder e eram auto suficientes, do mundo exterior. Hoje temos multiplas e variadas comunidades quilombolas no Brasil, que lutam para manter suas identidades cultural construida ao longo do tempo.\\

\hspace{1.5cm}
O quilombo Mesquita origina-se no século 18, com o final da mineração no Estado de Goias. Após mais de 200 anos conseguiu sua titulação junto ao INCRA e Fundação Palmares, como uma comunidade quilombola. Para que esta conquista atinja as gerações futuras suas tradições devem ser passada. Este observador procurou ver como estas tradições eram passada dentro desta sociedade, com vista a reforçar a identidade dos mais jovens como quilombolas. Dito isto, não foi assistido aula na referida escola e sim investigando junto a comunidade como este processo é realizado. Na comunidade o museu, localizado na antiga igreja, é um local que conta a história dos antepassados através dos objetos doados pelos diversos moradores. Os casarões antigos também contam suas historias, juntos com a estruturas das casas, que se transformaram em alvenaria, com seus quintais e hortas. A associação tem um projeto de memória viva, onde os conhecedores da cultura local, mais velhos, vão a escola passar seus conhecimentos aos mais jovens. As festas, durante o decorrer do ano, tais como as folias e a do doce de marmelo, também contribui para construção do ser quilombola nos seus integrantes. Para Oliveira, as manifestações religiosas africanas, também fazem parte da formação da identidade mas são menos aparentes devido ao sincretismo religioso.\\

\hspace{1.5cm}
Para Vygotsky\cite{marta}, o homem transformar-se de um ser biológico para sócio-histórico. Isto se dar pela cultura  que constrói a natureza do indivíduo ao longo da vida. Ao observar a interação se dar dentro da comunidade, temos no social o papel de desenvolvedor da cultura tradicional, como a manutenção da memória quilombola. Neste contexto o que se destaca é o projeto memória viva, pois nele carrega toda característica cultural, do lo local. Com ele é passado para toda a comunidade suas tradições e costumes pelos os detentores do conhecimento acumulado, no decorrer dos anos. Conhecimentos como a fabricação do doce de marmelo, das organização das folias (com suas cantigas), da confecção dos artesanatos, do açafrão e da cachaça da região. Aqui temos o papel da mediação, de Vygotsky, no processo de aprendizagem da cultura tradicional pelo integrantes mais jovens do quilombo. Esta mediação tem o papel de criar a identidade em toda população jovens da comunidade e assim reforçar tradições culturais. Segundo Wallon, a construção da personalidade do homem se dar na sociedade, ou seja no social. Ele também acreditava que o biológico tinha papel fundamental na formação do individuo. Neste sentido a função social e biológica do quilombo contribui para a formação da identidade cultural dos integrantes da comunidade.\\

\hspace{1.5cm}
Segundo o Ministério da Educação e Cultura (MEC), nas comunidades quilombolas no Brasil as unidades educacionais estão longes das residências dos alunos, as condições de estrutura são precárias e poucas comunidades têm unidades educacional com o ensino fundamental completo. Partindo destas questões, temos no quilombo Mesquita uma contrastação diferente do que acontece no nível nacional. Lá há uma escola com o nível fundamental completo, mais da metade do seu quadro de professores são da comunidade. A escola, junto com a sociedade local, entra neste contexto agente formador da identidade cultural dos ocupantes deste quilombo, principalmente os jovens. Pois, muito deles quando chegar no ensino médio (Brasília/DF, Cidade Ocidental/GO, Luziânia/GO e Valparaíso/GO), terão contato com outras cultural. Este novo mundo pode desestabilizar o quilombola, fazendo questionar sua identidade e qual a necessidade de resgate de suas memória. Os educadores tem o papel fundamental neste processo de autoafirmação, destes jovens, para que saibam de onde vêm e qual o horizonte deve se tornar, na compreensão de mundo. Que neste novo universo seu quilombo teve um importante papel na construção de Brasília/DF, principalmente na de madeira, com abastecimento de hortifrutigranjeiro. Estas frutas e verduras alimentavam os operários desta grandiosa empreitada. Outro fato que deve ser destacado é o grande apreço da família real para com o doce de marmelo como uma das iguarias de Goias.\\
\section*{Conclusão}
\hspace{1.5cm}
No mundo, com advento da tecnologia, todas as profissões/carreiras buscam o auto-aperfeiçoamento. Mas o sistema educacional continua utilizando os mesmos métodos, de fazer educação, baseado no passado. 
Esta concepção está definida na transformação da educação em mercadoria, onde gera e ao mesmo tempo é consumida por uma manipulação conteúdo. É preciso buscar uma sistema educacional com mas autonomia, dos professores e alunos. Este processo de autonomia deve ser conquista aos poucos, com a sociedade, alunos, professores e gestores juntos na busca do mesmo ideal. Dentro deste sistema o desenvolvimento educacional se dará dentro do convívio social-cultural entre alunos-alunos, alunos-professores, professores-professores e alunos-professores-sociedade, em um aprendizado continuo. Isto mostra que o aprendizado dar-se-á em todos os níveis de convivência,  sem que um seja melhor que o outro. Dentro desta perspectiva a educação deve saber explorar os saberes tradicionais, as histórias de vidas (professores, alunos e sociedade), pois tudo isto faz parte da formação da identidade de todos, bem como da própria escola. 